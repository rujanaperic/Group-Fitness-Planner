%\chapter{Zaključak i budući rad}
		
		%\textbf{\textit{dio 2. revizije}}\\
		
		{Primarni cilj ovog projekta bio je razviti aplikaciju koja omogućuje polaznicima treninga odabir treninga po terminu i tipu, koje korisnik može pohađati u skladu sa svojim slobodnim vremenom. Na taj način termini vježbanja prilagođavaju se slobodnom vremenu vježbača (polaznika), te su vježbači puno zadovoljniji jer mogu uložiti više pažnje vježbanju, i dobiti maksimalnu korist i dobrobit od usluga koje teretana, koja se koristi aplikacijom, pruža. \\}

        {Iz projektnog zadatka bilo je potrebno izlučiti funkcionalne i nefunkcionalne zahtjeve, konceptualno osmisliti, dokumentirati, a zatim i implementirati osmišljeno. Mentori ovog projekta ujedno su „glumili“ i klijente, pa smo komunikacijom s njima definirali koje funkcionalnosti žele vidjeti u aplikaciji, što ona mora sadržavati.  \\}

        {Provedba projekta bila je podijeljena u dva ciklusa. U prvom ciklusu oformili smo projektni tim, uspostavili kanale komunikacije, upoznali se sa zadatkom i krenuli u osmišljavanje rješenja zadatka. Inicijalne funkcionalnosti koje je naša aplikacija imala bili su prijava i registracija korisnika u sustav. Redovitim sastancima i komunikacijom preko društvenih mreža podijelili smo se u timove, zadužili se za pojedine zadatke i funkcionalnosti, te jasno definirali što koja funkcionalnost radi kako bismo svi imali jednaku ideju i razumjevanje zadaka. \\} 

        {U drugom ciklusu veći naglasak je bio na implementaciji same aplikacije. Dok smo se svi u prvom ciklusu „upoznavali“ i po prvi put susreli sa programskim alatima koje trebamo koristiti, u drugom ciklusu naglasak je bio na individualnom radu i programskoj implementaciji. Sve funkcionalnosti opisane obrascima uporabe su implementiranje, a one koje nisu opisane, nisu niti implementirane. \\}

        {Korist ovog projekta bila je prvenstveno edukacijska. Jako puno smo naučili o tehnologijama kojima smo implementirali aplikaciju, ali i o timskom radu, komunikaciji i dobili smo uvid od svojih mentora kako je raditi  takve projekte u pravim firmama za stvarne klijente. Iako smo se trudili implementirati sve željene funkcionalnosti, svjesni smo kako je potrebno još mnogo rada i vještine kako bi ona zaista bila na nivou na kakav su korisnici danas navikli. Trenutna verzija simbolizira prototip kojeg bilo koji član tima jednog dana može unaprjeđivati i usavršavati ukoliko bude imao interesa. Smatramo da je aplikacija korisna i primjenjiva svakom od nas jer se, kao i većina ljudi, brinemo o svome zdravlju, nastojimo baviti fizičkom aktivnosti i povremeno odlaziti u teretane, te se lako vidimo kako aplikaciju u budućnosti zaista i upotrebljavamo. \\}
		
		 %\textit{Potrebno je točno popisati funkcionalnosti koje nisu implementirane \\u ostvarenoj aplikaciji.}
		
		%\eject 